\documentclass{article}
\usepackage{xeCJK} % For CJK support
\usepackage[margin=1in]{geometry} % Set all margins to 1 inch
\usepackage{parskip}% Add more space

\usepackage{amsmath} % is needed to use the math environments align, aligned, gather, gathered, multline etc.
\usepackage{amsfonts} % For math fonts like \mathbb{R}
\usepackage{amsthm} % For theorem environments

\usepackage{cases} % For \begin{numcases} environment, included after amsmath
\usepackage{braket} % For \set{} command

\usepackage[hidelinks]{hyperref}
\newcommand{\0}{\mathit{0}}
\newcommand{\V}{\mathsf{V}}


\begin{document}
\section{Page 12}
\textbf{Theorem 1.2}. In any vector space $\V$, the following statements are true:

(a) $0x = \0$ for each $x \in \V$.

(b) $(-a)x = -(ax)=a(-x)$ for each $a \in F$ and each $x \in \V$.

(c) $a\0 = \0$ for each $a \in F$.

The proof of (c) is similar to the proof of (a)

\begin{quotation}
    \textbf{Theorem 1.1} (Cancellation Law for Vector Addition). If $x$, $y$,
    and $z$ are vectors in a vector space $\V$ such that $x+z = y +z$,then$x = y$.

    (\textbf{VS 7}) For each element $a$ in $F$ and each pair of elements $x$, $y$ in $\V$,
    $a(x +y)=ax+ay$. (See page 7)
\end{quotation}

\begin{proof}
    By VS 7, we have $a\0+a\0=a(\0+\0)=a\0=a\0+\0$.
    Then by the cancellation law for vector addition, we have $a\0 = \0$.
\end{proof}

\section{Thoughts on exercise 1.2.1(c)(d)}

\begin{quotation}

    1. Label the following statements as true or false.

    (c)In any vector space, $ax = bx$ implies that $a = b$.

    (d)In any vector space, $ax = ay$ implies that $x = y$.

    (\textbf{VS 5}) For each element $x$ in $\V$, $1x = x$.

    (\textbf{VS 6}) For each pair of elements $a$, $b$ in $F$ and each element $x$ in $\V$,
    $(ab)x = a(bx)$.

\end{quotation}

It's obvious that in (c) and (d) above, $0$ and $\0$ is a counterexample respectively. Then:

If $ax=\0\ (a\in F,x\in\V)$, is it possible that $a\neq0$ and $v\neq\0$?

\begin{proof}[Answer]
    No, it's impossible.

    Since $a$ is an element in field,
    so there's multiplicative inverse $a^{-1}$ such that $a^{-1} a=1$.

    By VS 6 and VS 5, $a^{-1} (ax)=(a^{-1}a)x=1x=x$, also $a^{-1} (ax)=a^{-1}\0=0$.
\end{proof}

\section{Thoughts on Exercise 1.3.28}
Why is the condition "$F$ is not of characteristic $2$" needed?

\begin{quotation}
    28. A matrix $M$ is called \textit{skew-symmetric} if $M^t = -M$.
    Clearly, a skew-symmetric matrix is square. Let $F$ be a field. Prove that the set $\mathsf{W}_1$ of all skew-symmetric $n\times n$ matrices with entries from F is a subspace of $\mathsf{M}_{n\times n}(F)$. Now assume that $F$ is not of characteristic 2 (see Ap
    pendix C), and let $\mathsf{W}_2$ be the subspace of $\mathsf{M}_{n\times n}(F)$ consisting of all
    symmetric $n\times n$ matrices. Prove that$\mathsf{M}_{n\times n}(F)=\mathsf{W}_1 \oplus \mathsf{W}_2$.


    \textbf{Example 4}
    The field $Z_2$ consists of two elements $0$ and $1$ with the operations of addition
    and multiplication defined by the equations
    \begin{gather*}
        0+0=0,\quad 0+1=1+0=1,\quad1+1=0,\\
        0\cdot 0=0, \quad 0\cdot 1=1\cdot 0=0, \quad \text{and} \quad1\cdot 1=1
    \end{gather*}

    In an arbitrary field $F$, it may happen that a sum $1+1+\cdots +1$($p$ sum
    mands) equals $0$ for some positive integer $p$. For example, in the field $Z_2$
    (defined in Example 4), $1+1 = 0$. In this case, the smallest positive integer $p$
    for which a sum of $p$ $1$’s equals $0$ is called the \textbf{characteristic} of $F$; if no such
    positive integer exists, then F is said to have characteristic zero.Thus $Z_2$
    has characteristic two, and $R$ has characteristic zero. Observe that if $F$ is a
    field of characteristic $p=0$,then $x+x+\cdots +x$ ($p$ summands) equals $0$ for all
    $x \in F$.

\end{quotation}

\begin{proof}
    The solution mentioned
    \begin{quotation}
        $\mathsf{W}_1 \cap\mathsf{W}_2 ={\0}$ holds is because $A+A^t$ is symmetric and $A-A^t$ is skew-symmetric. If $F$ is of characteristic 2, we have
        $\mathsf{W}_1  =\mathsf{W}_2 $
    \end{quotation}

    if there's other element $A$ other than $\0$ in $\mathsf{W}_1 \cap\mathsf{W}_2$, then there is non-zero entry $A_{ij}=A_{ji}$ and $A_{ij}=-A_{ji}$, which leads to
    $A_{ij}+A_{ij}=0$.

    Devide both hands of equity by $A_{ij}$ causes $1+1=0$, which means $F$ is of characteristic $2$”
\end{proof}

\section{Thoughts on exercise 1.5.13}
Where is the condition "characteristic not equal to two" used?

\begin{quotation}
    13. Let $\V$ be a vector space over a field of characteristic
    not equal to two.

    (a) Let $u$ and $v$ be distinct vectors in $V$. Prove that $\set{u, v}$ is
    linearly independent if and only if $\set{u + v, u - v}$ is linearly independent.

    (b) Let $u, v,$ and $w$ be distinct vectors in $\V$. Prove that $\set{u, v, w}$
    is linearly independent if and only if $\set{u + v, u + w, v + w}$ is linearly
    independent.
\end{quotation}

Since the proof of (a) and (b) are similar,
we only show where the characteristic not equal to two is used in the sufficiency case of (a).
For the rest, the situation is similar.

\begin{proof}
    If $\set{u+v, u-v}$ is linearly independent, we have
    $m(u+v) + n(u - v) = \0 \implies m = n = \0$.
    Assume $xu+yv=0$, then we try to represent $xu+yv$ as linear combination of $u+v$ and $u-v$.
    Let $a(u+v)+b(u-v)=xu+yv$, then we solve the system of equations:
    \begin{numcases}{}
        a + b = x \tag*{(1)}\\
        a - b = y \tag*{(2)}
    \end{numcases}
    By adding (1) and (2), and using distributivity of multiplication over addition
    we have $\mathit{2}a = x + y$. Note that $\mathit{2}$ here is just a notation of $1+1$ in the field.
    \textbf{Since field is not of characteristic $2$, $\mathit{2}$ is not equal to $0$.}
    Then there exists a multiplicative inverse $\mathit{2}^{-1}$ in $F$.
    So by multiplying $\mathit{2}^-1$ on both side,
    we have $a = \frac{x+y}{\mathit{2}}$ and $b = \frac{x-y}{\mathit{2}}$.
    Thus we have
    \begin{align*}
        xu+yv & =a(u+v)+b(u-v)
              & =\frac{x+y}{\mathit{2}}(u+v)+\frac{x-y}{\mathit{2}} \\
    \end{align*}
    By the linear independence of $\set{u+v, u-v}$, we have
    \begin{numcases}{}
        \frac{x+y}{\mathit{2}} = 0 \tag*{(3)}\\
        \frac{x-y}{\mathit{2}} = 0 \tag*{(4)}
    \end{numcases}
    Similarly, thanks to field's characteristic is not equal to two,
    we can first multiply $\mathit{2}$ on both side of (3) and (4),
    then we can add (3) and (4) to get $x = 0$ and $y = 0$.
\end{proof}


\section{Note exercise 1.5.15}
\begin{quotation}
    15. Let $S = \set{u_1, u_2, \dots, u_n}$ be a finite set of vectors.
    Prove that $S$ is linearly dependent if and only if $u_1 = 0$ or
    $u_{k+1} \in \operatorname{span}(\set{u_1, u_2, \dots, u_k})$ for some
    $k$ such that $1 \leq k < n$.
\end{quotation}

The wording of the exercise is not so clear.
I think for the "$\implies$" part, it should be
\begin{quote}
    If $S = \set{u_1, u_2, \dots, u_n}$ is a linearly dependent set, then
    $\exists k$ satisfying $1 \leq k < n$ such that
    $u_{k+1} \in \operatorname{span}(\set{u_1, u_2, \dots, u_{k-1}})$.
\end{quote}
Note it's $\exists k$ rather than $\forall k$.

If it's $\forall k$, then the statement is not true. For example,
let $S$ be ${u_1,u_2,u_3}$, where
$u_1=(1,0)$,$u_2=(2,0)$ and $u_3=(0,1)$.
Clearly $S$ is linearly independent, but $u_3$ is not in the span of $\set{u_1,u_2}$.

Besides, I don't think the proof of "$\implies$" in the solution book is clear.

The following only shows the "$\implies$" part,
since the "$\impliedby$"
is simple and clearly showed in the solution book.

\begin{quotation}
    \textbf{Theorem 1.7.} Let $S$ be a linearly independent subset of a vector space
    $V$, and let $v$ be a vector in $V$ that is not in $S$. Then $S \cup \set{v}$
    is linearly dependent if and only if $v \in \operatorname{span}(S)$.
\end{quotation}

\begin{proof}
    Let induct on $n$.

    For $n=1$, we have $S=\set{u_1}$. If $S$ is linearly dependent, $u_1$ have to be $\0$.

    If the statement is true for $n$, we show it is also true for $n+1$.
    Since $\set{u_1,u_2,\dots,u_{n+1}}=\set{u_1,u_2,\dots,u_n}\cup\set{u_{n+1}}$
    If $\set{u_1,u_2,\dots,u_n}$ is linearly dependent, then by induction hypothesis,
    it's already true.
    If $\set{u_1,u_2,\dots,u_n}$ is linearly independent, then by Theorem 1.7, we get
    $u_n\in\operatorname{span}(\set{u_1,u_2,\dots,u_n})$

    Then finish the proof by induction.
\end{proof}

\section{Takeaway}

\textbf{Theorem 1.3.}
Let $\mathsf{V}$ be a vector space and $W$ a subset of
$\mathsf{V}$. Then $W$ is a subspace of $\mathsf{V}$ if and
only if the following three conditions hold for the operations
defined in $\mathsf{V}$:

(a) $\0 \in W$.

(b) $x + y \in W$ whenever $x \in W$ and $y \in W$.

(c) $c x \in W$ whenever $c \in F$ and $x \in W$.

\medskip

\textbf{Definition.}
Let $\mathsf{V}$ be a vector space and $\set{S}$ a nonempty subset of
$\mathsf{V}$. A vector $v \in \mathsf{V}$ is called a \textbf{linear combination}
of vectors of $\set{S}$ if there exist a finite number of vectors
$u_1, u_2, \dots, u_n$ in $\set{S}$ and scalars $a_1, a_2, \dots, a_n$
in $F$ such that
\[
    v = a_1 u_1 + a_2 u_2 + \dots + a_n u_n.
\]
In this case, we also say that $v$ is a linear combination of
$u_1, u_2, \dots, u_n$ and call $a_1, a_2, \dots, a_n$ the \textbf{coefficients}
of the linear combination.

\medskip

\textbf{Definition.}
A subset $\set{S}$ of a vector space $\mathsf{V}$ is called \textbf{linearly
    dependent} if there exist a finite number of distinct vectors
$u_1, u_2, \dots, u_n$ in $\set{S}$ and scalars $a_1, a_2, \dots, a_n$,
not all zero, such that
\[
    a_1 u_1 + a_2 u_2 + \dots + a_n u_n = \0.
\]
In this case, we also say that the vectors of $\set{S}$ are linearly dependent.

\medskip

\textbf{Definition.}
A subset $\set{S}$ of a vector space $\mathsf{V}$ that is not linearly
dependent is called \textbf{linearly independent}.

As before, we also say that the vectors of $\set{S}$ are linearly independent.

\medskip

\textbf{Definition.}
A \textbf{basis} $\beta$ for a vector space $\mathsf{V}$ is a linearly independent
subset of $\mathsf{V}$ that generates $\mathsf{V}$.

If $\beta$ is a basis for $\mathsf{V}$, we also say that the vectors of
$\beta$ form a basis for $\mathsf{V}$.

\end{document}